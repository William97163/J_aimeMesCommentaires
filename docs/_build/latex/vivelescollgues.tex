%% Generated by Sphinx.
\def\sphinxdocclass{report}
\documentclass[letterpaper,10pt,french]{sphinxmanual}
\ifdefined\pdfpxdimen
   \let\sphinxpxdimen\pdfpxdimen\else\newdimen\sphinxpxdimen
\fi \sphinxpxdimen=.75bp\relax
\ifdefined\pdfimageresolution
    \pdfimageresolution= \numexpr \dimexpr1in\relax/\sphinxpxdimen\relax
\fi
%% let collapsible pdf bookmarks panel have high depth per default
\PassOptionsToPackage{bookmarksdepth=5}{hyperref}

\PassOptionsToPackage{warn}{textcomp}
\usepackage[utf8]{inputenc}
\ifdefined\DeclareUnicodeCharacter
% support both utf8 and utf8x syntaxes
  \ifdefined\DeclareUnicodeCharacterAsOptional
    \def\sphinxDUC#1{\DeclareUnicodeCharacter{"#1}}
  \else
    \let\sphinxDUC\DeclareUnicodeCharacter
  \fi
  \sphinxDUC{00A0}{\nobreakspace}
  \sphinxDUC{2500}{\sphinxunichar{2500}}
  \sphinxDUC{2502}{\sphinxunichar{2502}}
  \sphinxDUC{2514}{\sphinxunichar{2514}}
  \sphinxDUC{251C}{\sphinxunichar{251C}}
  \sphinxDUC{2572}{\textbackslash}
\fi
\usepackage{cmap}
\usepackage[T1]{fontenc}
\usepackage{amsmath,amssymb,amstext}
\usepackage{babel}



\usepackage{tgtermes}
\usepackage{tgheros}
\renewcommand{\ttdefault}{txtt}



\usepackage[Sonny]{fncychap}
\ChNameVar{\Large\normalfont\sffamily}
\ChTitleVar{\Large\normalfont\sffamily}
\usepackage{sphinx}

\fvset{fontsize=auto}
\usepackage{geometry}


% Include hyperref last.
\usepackage{hyperref}
% Fix anchor placement for figures with captions.
\usepackage{hypcap}% it must be loaded after hyperref.
% Set up styles of URL: it should be placed after hyperref.
\urlstyle{same}

\addto\captionsfrench{\renewcommand{\contentsname}{Contenu :}}

\usepackage{sphinxmessages}
\setcounter{tocdepth}{1}



\title{viveLesCollègues}
\date{janv. 06, 2023}
\release{}
\author{william}
\newcommand{\sphinxlogo}{\vbox{}}
\renewcommand{\releasename}{}
\makeindex
\begin{document}

\ifdefined\shorthandoff
  \ifnum\catcode`\=\string=\active\shorthandoff{=}\fi
  \ifnum\catcode`\"=\active\shorthandoff{"}\fi
\fi

\pagestyle{empty}
\sphinxmaketitle
\pagestyle{plain}
\sphinxtableofcontents
\pagestyle{normal}
\phantomsection\label{\detokenize{index::doc}}



\chapter{vivelescollegues\_william}
\label{\detokenize{modules:vivelescollegues-william}}\label{\detokenize{modules::doc}}

\section{Contexte}
\label{\detokenize{contexte:contexte}}\label{\detokenize{contexte::doc}}

\subsection{Contexte du projet}
\label{\detokenize{contexte:contexte-du-projet}}\phantomsection\label{\detokenize{contexte:module-contexte.contexte}}\index{module@\spxentry{module}!contexte.contexte@\spxentry{contexte.contexte}}\index{contexte.contexte@\spxentry{contexte.contexte}!module@\spxentry{module}}
\sphinxAtStartPar
Vous êtes responsable de la mise à jour de la liste des employés de votre entreprise. 
Il s’agit d’une liste enregistrée dans un fichier .txt, et disponible sur un serveur Gitlab distant. 
A chaque nouvelle arrivée, vous devez ajouter le nouvel employé dans cette liste.


\section{Données}
\label{\detokenize{datas:donnees}}\label{\detokenize{datas::doc}}

\subsection{Fichier de départ}
\label{\detokenize{datas:fichier-de-depart}}\phantomsection\label{\detokenize{datas:module-datas.explications}}\index{module@\spxentry{module}!datas.explications@\spxentry{datas.explications}}\index{datas.explications@\spxentry{datas.explications}!module@\spxentry{module}}
\sphinxAtStartPar
Le tableau ci\sphinxhyphen{}dessous correspond au fichier csv dont vous disposez :


\begin{savenotes}\sphinxattablestart
\centering
\begin{tabulary}{\linewidth}[t]{|T|T|T|T|}
\hline

\sphinxAtStartPar
nom
&
\sphinxAtStartPar
Prénom
&
\sphinxAtStartPar
Date d’embauche
&
\sphinxAtStartPar
Poste
\\
\hline
\sphinxAtStartPar
Roberts
&
\sphinxAtStartPar
Julia
&
\sphinxAtStartPar
10/10/2022
&
\sphinxAtStartPar
Patronne
\\
\hline
\sphinxAtStartPar
Doe
&
\sphinxAtStartPar
John
&
\sphinxAtStartPar
11/10/2022
&
\sphinxAtStartPar
Ingénieur
\\
\hline
\sphinxAtStartPar
Depp
&
\sphinxAtStartPar
Johnny
&
\sphinxAtStartPar
11/10/2022
&
\sphinxAtStartPar
Concierge
\\
\hline
\sphinxAtStartPar
Parker
&
\sphinxAtStartPar
Peter
&
\sphinxAtStartPar
01/10/2022
&
\sphinxAtStartPar
Stagiaire
\\
\hline
\sphinxAtStartPar
Sylvester
&
\sphinxAtStartPar
Stalonne
&
\sphinxAtStartPar
15/10/2022
&
\sphinxAtStartPar
Manager
\\
\hline
\sphinxAtStartPar
Winnie
&
\sphinxAtStartPar
l’Ourson
&
\sphinxAtStartPar
22/10/2022
&
\sphinxAtStartPar
Cuisinier
\\
\hline
\sphinxAtStartPar
Harry
&
\sphinxAtStartPar
Potter
&
\sphinxAtStartPar
25/10/2022
&
\sphinxAtStartPar
Ingénieur R\&D
\\
\hline
\sphinxAtStartPar
Andraud
&
\sphinxAtStartPar
Laurence
&
\sphinxAtStartPar
01/10/2022
&
\sphinxAtStartPar
Alternante
\\
\hline
\end{tabulary}
\par
\sphinxattableend\end{savenotes}


\section{Script}
\label{\detokenize{add_new_employees:script}}\label{\detokenize{add_new_employees::doc}}

\subsection{fonctions du script}
\label{\detokenize{add_new_employees:module-add_new_employees.main}}\label{\detokenize{add_new_employees:fonctions-du-script}}\index{module@\spxentry{module}!add\_new\_employees.main@\spxentry{add\_new\_employees.main}}\index{add\_new\_employees.main@\spxentry{add\_new\_employees.main}!module@\spxentry{module}}
\sphinxAtStartPar
Voici la documentation du fichier main.py, ici vous retrouverez
\begin{itemize}
\item {} 
\sphinxAtStartPar
Toutes les fonctions de ce fichier

\item {} 
\sphinxAtStartPar
Les détails les expliquants

\end{itemize}

\sphinxAtStartPar
Ce script a pour fonction principale de faire apparaître une application permettant d’insérer un employé par son nom, son prénom,
sa date d’embauche ainsi que son poste
\index{btn\_clicked() (dans le module add\_new\_employees.main)@\spxentry{btn\_clicked()}\spxextra{dans le module add\_new\_employees.main}}

\begin{fulllineitems}
\phantomsection\label{\detokenize{add_new_employees:add_new_employees.main.btn_clicked}}\pysiglinewithargsret{\sphinxcode{\sphinxupquote{add\_new\_employees.main.}}\sphinxbfcode{\sphinxupquote{btn\_clicked}}}{}{}
\sphinxAtStartPar
Cette fonction permet d’ajouter un employé dans le fichier : « liste\_employes.txt »

\end{fulllineitems}

\index{error\_popup() (dans le module add\_new\_employees.main)@\spxentry{error\_popup()}\spxextra{dans le module add\_new\_employees.main}}

\begin{fulllineitems}
\phantomsection\label{\detokenize{add_new_employees:add_new_employees.main.error_popup}}\pysiglinewithargsret{\sphinxcode{\sphinxupquote{add\_new\_employees.main.}}\sphinxbfcode{\sphinxupquote{error\_popup}}}{}{}
\sphinxAtStartPar
Cette fonction fait apparaitre une popup si l’utilisateur oubli de rentrer un argument ou plusieurs arguments
\begin{description}
\item[{Args:}] \leavevmode
\sphinxAtStartPar
Pas d’arguments

\item[{Returns:}] \leavevmode
\sphinxAtStartPar
Arguments Missing.

\end{description}

\end{fulllineitems}

\index{validate\_popup() (dans le module add\_new\_employees.main)@\spxentry{validate\_popup()}\spxextra{dans le module add\_new\_employees.main}}

\begin{fulllineitems}
\phantomsection\label{\detokenize{add_new_employees:add_new_employees.main.validate_popup}}\pysiglinewithargsret{\sphinxcode{\sphinxupquote{add\_new\_employees.main.}}\sphinxbfcode{\sphinxupquote{validate\_popup}}}{}{}
\sphinxAtStartPar
Cette fonction fait apparaitre une popup si l’utilisateur rentre tous les arguments demandés
\begin{description}
\item[{Args:}] \leavevmode
\sphinxAtStartPar
Pas de paramètre

\item[{Returns:}] \leavevmode
\sphinxAtStartPar
Task successful.

\end{description}

\end{fulllineitems}



\chapter{Indices and tables}
\label{\detokenize{index:indices-and-tables}}\begin{itemize}
\item {} 
\sphinxAtStartPar
\DUrole{xref,std,std-ref}{genindex}

\item {} 
\sphinxAtStartPar
\DUrole{xref,std,std-ref}{modindex}

\item {} 
\sphinxAtStartPar
\DUrole{xref,std,std-ref}{search}

\end{itemize}


\renewcommand{\indexname}{Index des modules Python}
\begin{sphinxtheindex}
\let\bigletter\sphinxstyleindexlettergroup
\bigletter{a}
\item\relax\sphinxstyleindexentry{add\_new\_employees.main}\sphinxstyleindexpageref{add_new_employees:\detokenize{module-add_new_employees.main}}
\indexspace
\bigletter{c}
\item\relax\sphinxstyleindexentry{contexte.contexte}\sphinxstyleindexpageref{contexte:\detokenize{module-contexte.contexte}}
\indexspace
\bigletter{d}
\item\relax\sphinxstyleindexentry{datas.explications}\sphinxstyleindexpageref{datas:\detokenize{module-datas.explications}}
\end{sphinxtheindex}

\renewcommand{\indexname}{Index}
\printindex
\end{document}